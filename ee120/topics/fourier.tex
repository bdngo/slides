\section{Fourier Transforms}

\begin{frame}{Continuous Time Fourier Transform}
    \begin{itemize}
        \item Given a continuous-time signal $x(t)$, you can apply the \textbf{Continuous Time Fourier Transform} (CTFT) to find its \textbf{spectrum}, $X(\omega)$. 
        \item $X(\omega)$ is also known as the \textbf{frequency domain representation} of $x(t)$.
        \item The CTFT (sometimes known as the \textbf{CTFT analysis equation}) can be written as follows:
        \begin{align*}
            X(\omega) = \int_{-\infty}^\infty x(t) e^{-i\omega t}\, dt. 
        \end{align*}
    \end{itemize}
\end{frame}

\begin{frame}{CTFT (Continued)}
    \begin{itemize}
        \item Given the spectrum of a signal, you can find the time-domain representation using the inverse CTFT (also known as the \textbf{CTFT synthesis equation}):
        \begin{align*}
            x(t) = \frac{1}{2\pi} \int_{-\infty}^\infty X(\omega) e^{i\omega t}\, d\omega. 
        \end{align*}
        \item If $h(t)$ is the impulse response of LTI system $\mathcal{H}$, the CTFT of $h(t)$, $H(\omega)$ is known as the \textbf{frequency response} of $\mathcal{H}$. 
        \begin{itemize}
            \item For any input signal of the form $x(t) = e^{i\omega t}$, the output of the system will be $y(t) = H(\omega) x(t)$.
            \item We sometimes call such an $x(t)$ an \textbf{eigenfunction} of $\mathcal{H}$.
        \end{itemize}
    \end{itemize}
\end{frame}

\begin{frame}{CTFT (Continued)}
    \begin{itemize}
        \item You can only use the CTFT analysis equation if $x(t)$ is absolutely integrable, i.e.
        \begin{align*}
            \int_{-\infty}^\infty |x(t)| \, dt < \infty.
        \end{align*}
        \item For signals that are not absolutely integrable but have a CTFT representation, you can find the spectrum using other methods.
        \begin{itemize}
            \item For periodic signals, you can calculate the CTFS and convert from the CTFS to the CTFT.
            \item Later on, you will learn about properties of Fourier Transforms you can utilize.
        \end{itemize}
    \end{itemize}
\end{frame}

\begin{frame}{Practice: CTFT}
    \begin{enumerate}
        \item Find the CTFT of the signal
        $x(t) = e^{t} (u(t + 3) - u(t - 3)).$
        \item Given that the CTFT of $y(t)$ is $Y(\omega) = \delta(\omega - \pi)$, find $y(t)$. 
    \end{enumerate}
\end{frame}

\begin{frame}{Practice: CTFT (Solutions)}
    \begin{enumerate}
        \item {\color{red} $X(\omega) = \frac{2i}{1 - i\omega} sin(3 - 3i\omega).$}
        {\color{blue}
        \begin{alignat*}{2}
            X(\omega) &= \int_{-\infty}^\infty x(t) e^{-i\omega t}\, dt
            &&= \int_{-3}^3 e^{t}e^{-i\omega t}\, dt \\
            &= \int_{-3}^3 e^{(1 - i\omega)t}\,dt 
            &&= \frac{1}{1 - i\omega} e^{1 - i\omega t} \bigg\rvert_{-3}^3  \\
            &= \frac{1}{1 - i\omega} \big(e^{3 - 3i\omega} - e^{-(3 - 3i\omega)}\big)
            &&= \frac{2i}{1 - i\omega} \sin(3 - 3i\omega).
        \end{alignat*}
        }
        \item {\color{red} $y(t) = e^{i\pi t}.$}
    \end{enumerate}
\end{frame}

\begin{frame}{Practice: CTFT (Solutions)}
    \begin{enumerate}
        \item {\color{red} $X(\omega) = \frac{2i}{1 - i\omega} sin(3 - 3i\omega).$}
        \item {\color{red} $y(t) = e^{i\pi t}.$}
        {\color{blue}
        \begin{alignat*}{2}
            y(t) &= \frac{1}{2\pi} \int_{-\infty}^{\infty} Y(\omega) e^{i\omega t} \, d\omega 
            &&= \frac{1}{2\pi} \int_{-\infty}^{\infty} \delta(\omega - \pi)  e^{i\omega t} \, d\omega \\
            &= \frac{1}{2\pi} \int_{-\infty}^{\infty} \delta(\omega - \pi)  e^{i\pi t} \, d\omega
            &&= \frac{1}{2\pi} e^{i\pi t}.
        \end{alignat*}
        }
    \end{enumerate}
\end{frame}

\begin{frame}{Continuous Time Fourier Series}
    \begin{itemize}
        \item For periodic continuous-time signals (where $x(t) = x(t + T),\, \forall t$), we can represent $x(t)$ as a sum of complex exponentials at multiples of the fundamental frequency $\omega_0 = \frac{2\pi}{T}$.
        \item The CTFS synthesis equation is as follows:
        \begin{align*}
            x(t) = \sum_{k = -\infty}^\infty X_k e^{i \frac{2\pi}{T}kt},
        \end{align*}
        where $X_k$ is the $k^{\text{th}}$ \textbf{Fourier coefficient} of $x(t)$.
    \end{itemize}
\end{frame}

\begin{frame}{CTFS (Continued)}
    \begin{itemize}
        \item $X_k$ is can be calculated as follows (CTFS analysis equation):
        \begin{align*}
            X_k = \frac{1}{T}\int_{<T>} x(t) e^{-i \frac{2\pi}{T}kt}\, dt.
        \end{align*}
        Note that $\int_{<T>}$ denotes an integral over any interval of length $T$ (for instance, $0$ to $T$ or $-T/2$ to $T/2$).
        \item You can convert from the CTFS to the CTFT as follows:
        \begin{align*}
            X(\omega) = 2\pi \sum_{k = -\infty}^\infty X_k \delta(\omega - \frac{2\pi k}{T}).
        \end{align*}
    \end{itemize}
\end{frame}

\begin{frame}{Finding CTFS Coefficients}
    \begin{itemize}
        \item Determine the period, $T$, and fundamental frequency $\omega_0 = \frac{2\pi}{T}$ of the signal.
        \item If you can write the signal as a sum of complex exponentials (for example, if it's a $\cos$ or a $\sin$), pattern-match the sum with the CTFS synthesis equation.
        \item Otherwise, use the CTFS analysis equation to find expressions for the Fourier coefficients.
    \end{itemize}
\end{frame}

\begin{frame}{Practice: CTFS}
    \begin{enumerate}
        \item Find the fundamental frequency and nonzero CTFS coefficients of $x(t) = \sin(\frac{\pi}{2}t) + \cos(2\pi t)$.
        \item Find the fundamental frequency and nonzero CTFS coefficients of $y(t) = \sum_{n = -\infty}^\infty \delta(t - n)$.
    \end{enumerate}
\end{frame}

\begin{frame}{Practice: CTFS (Solutions)}
    \begin{enumerate}
        \item {\color{red} $\omega_0 = \pi / 2$, $X_{-4} = \frac{1}{2}$, $X_-1 = \frac{-1}{2i}$, $X_1 = \frac{1}{2i}$, $X_4 = \frac{1}{2}$.} \\
        
        {\color{blue}
        \noindent First find the period of $x(t)$:
        $$x(t + T) = \sin(\frac{\pi}{2}(t + T)) + \cos(2\pi (t + T)) = x(t).$$
        The fundamental period of $x(t)$ is the smallest $T$ such that both $\frac{\pi}{2}T$ and $2\pi T$ are integer multiples of $2\pi$, so $T = 4$.

        $$\omega_0 = \frac{2\pi}{T} = \frac{\pi}{2}.$$
        }
    \end{enumerate}
\end{frame}

\begin{frame}{Practice: CTFS (Solutions)}
    \begin{enumerate}
        \item {\color{red} $\omega_0 = \pi / 2$, $X_{-4} = \frac{1}{2}$, $X_-1 = \frac{-1}{2i}$, $X_1 = \frac{1}{2i}$, $X_4 = \frac{1}{2}$.} \\
        
        {\color{blue}
        Now, we can write out $x(t)$ using Euler's formula and pattern-match with the CTFS synthesis equation.
        \begin{flalign*}
            x(t) &= \frac{1}{2i}\big(e^{\frac{\pi}{2} t} - e^{-\frac{\pi}{2} t} \big) + \frac{1}{2}\big(e^{2\pi t} + e^{-2\pi t} \big) \\
            &= \frac{1}{2i}\big(e^{\frac{\pi}{2} t} - e^{-\frac{\pi}{2} t} \big) + \frac{1}{2}\big(e^{\frac{\pi}{2}4t} + e^{-\frac{\pi}{2}4t} \big)
        \end{flalign*}
        Pattern-matching with
        \begin{align*}
            x(t) = \sum_{k = -\infty}^\infty X_k e^{i \frac{\pi}{2}kT},
        \end{align*}
        we can see that $X_{-4} = \frac{1}{2}$, $X_-1 = \frac{-1}{2i}$, $X_1 = \frac{1}{2i}$, $X_4 = \frac{1}{2}$, and the rest of the CTFS coefficients are $0$.
        }
        \item {\color{red} $\omega_0 = 2\pi$, $X_k = 1,\,\forall k$.}
    \end{enumerate}
\end{frame}

\begin{frame}{Practice: CTFS (Solutions)}
    \begin{itemize}
        \item {\color{red} $\omega_0 = \pi / 2$, $X_{-4} = \frac{1}{2}$, $X_-1 = \frac{-1}{2i}$, $X_1 = \frac{1}{2i}$, $X_4 = \frac{1}{2}$.}
        \item {\color{red} $\omega_0 = 2\pi$, $Y_k = 1,\,\forall k$.} \\
        
        {\color{blue}
        \noindent$y(t)$ repeats itself every $1$ timestep, so it has a period $T = 1$ and fundamental frequency $\omega_0 = 2\pi$. \\

        \noindent $y(t)$ cannot be easily written as a sum of sinusoids, so let's plug $y(t)$ into the CTFS analysis equation. Note that only one Dirac delta appears every period of the signal.
        \begin{flalign*}
            Y_k &= \frac{1}{T} \int_{-T/2}^{T/2} y(t) e^{-i\frac{2\pi}{T} t}\, dt \\
            &= \int_{-1/2}^{1/2} \delta(t) e^{-i 2\pi t}\, dt = e^{-i 2\pi (0)} = 1.
        \end{flalign*}
        }
    \end{itemize}
\end{frame}

% \begin{frame}{Discrete Time Frequency}
%     \begin{itemize}
%         \item Unlike in continuous time, discrete-time frequencies are periodic with a period of $2\pi$.
%         \item To see why, look at a discrete-time complex exponential with frequency $\omega + 2\pi$ for some $\omega$: 
%         \begin{align*}
%             e^{i(\omega + 2\pi) n} = e^{i\omega n}e^{i 2\pi n}.
%         \end{align*}
%         $e^{i2\pi n} = 1$ for integer $n$, so  
%         $$e^{i(\omega + 2\pi) n} = e^{i\omega n}.$$
%         \item In discrete time, we say that ``higher frequencies'' are closer to odd multiples of $\pi$ and ``lower frequencies'' are closer to even multiples of $\pi$.
%     \end{itemize}
% \end{frame}

% \begin{frame}{Discrete Time Fourier Transform}
%     \begin{itemize}
%         \item The spectrum of a discrete-time signal can be found using the \textbf{Discret Time Fourier Transform} (DTFT) / DTFT analysis equation:
%         \begin{align*}
%             X(\omega) = \sum_{n=-\infty}^\infty x[n] e^{-i\omega n}. 
%         \end{align*}
%         \item Given the spectrum of a signal, you can find the time-domain representation using the inverse DTFT / DTFT synthesis equation:
%         \begin{align*}
%             x[n] = \frac{1}{2\pi} \int_{<2\pi>} X(\omega) e^{i\omega n}\, d\omega. 
%         \end{align*}
%         \item Analogously to continuous time, you can only use the analysis equation if $x(t)$ is absolutely summable:
%         $$\sum_{n = -\infty}^{\infty} |x[n]| < \infty$$.
%         \item As in continuous time, the frequency response, $H(\omega)$, of an LTI system with impulse response $h[n]$ is provided by the DTFT of $h[n]$.
%     \end{itemize}
% \end{frame}

% \begin{frame}{Practice: DTFT}

% \end{frame}

% \begin{frame}{Discrete Time Fourier Series}
%     \begin{itemize}
%         \item For periodic discrete-time signals (where $x[n] = x[n + N],\, \forall t$ and an \textbf{integer value of N}), we can represent $x[n]$ as a sum of complex exponentials at multiples of the fundamental frequency $\omega_0 = \frac{2\pi}{N}$  (DTFS synthesis equation). 
%         \begin{align*}
%             x[n] = \sum_{k \in <N>} X_k e^{i \frac{2\pi}{N}kn}.
%         \end{align*}
%         \item Note that we are only summing over an interval of length N!
%         \item $X_k$ is can be calculated as follows (DTFS analysis equation):
%         \begin{align*}
%             X_k = \frac{1}{N}\sum{n \in <N>} x[n] e^{-i \frac{2\pi}{T}kn}\, dt.
%         \end{align*}
%         \item The DTFS coefficients are $N-$periodic, i.e. $X_{k} = X_{k + N}$.
%         \item You can convert from the DTFS to the DTFT as follow:
%         \begin{align*}
%             X(\omega) = 2\pi \sum_{k = -\infty}^\infty X_k \delta(\omega - \frac{2\pi k}{N}).
%         \end{align*}
%     \end{itemize}
% \end{frame}

% \begin{frame}{Practice: DTFS}

% \end{frame}

\begin{frame}{Frequency Domain and Convolution}
    \begin{itemize}
        \item \textbf{Convolution} in the time domain corresponds to \textbf{multiplication} in the frequency domain.
        \item If $y(t) = x(t) * h(t)$ in the time domain, then $Y(\omega) = X(\omega)H(\omega)$ in the CTFT domain.
        \item Likewise, multiplication in the time domain corresponds to convolution in the frequency domain.
        \item If $y(t) = x(t)h(t)$, then $Y(\omega) = X(\omega) * H(\omega)$.
    \end{itemize}
\end{frame}