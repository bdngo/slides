\section{Minimum Energy Control}

\begin{frame}{Controllability Recap}
	Recall that any system with state equation
	\[
		\vec{x}[n+1] = A\vec{x}[n] + B\vec{u}[n]
	\]
   	is controllable if and only if the controllability matrix
    \[
        C = 
        \begin{bmatrix}
        B & AB & A^2B & \cdots & A^{n-1}B
        \end{bmatrix}
    \]
    is full rank. For simplicity, we shall only consider the case where $u[t]$ is a scalar, so $B$ is a vector and $C$ is a full rank $n \times n$ matrix.
\end{frame}

\begin{frame}{Controllability Recap}
	If we wanted to arrive at our desired state $\vec{x}^*$ in $n$ timesteps, then we can simply choose
	\[
		\vec{u} 
		=
		\begin{bmatrix}
			u[n-1] \\
			u[n-2] \\
			\vdots \\
			u[0] \\
		\end{bmatrix}
		=
		C^{-1}\vec{x}^*
	\]
\end{frame}

\begin{frame}{Minimum Energy Control}
	If we want to arrive in $t > n$ timestamps, then there may be multiple solutions. The corresponding controllability matrix becomes
    \[
        \mathscr{C} = 
        \begin{bmatrix}
        B & AB & A^2B & \cdots & A^{t-1}B
        \end{bmatrix}
    \]
	One natural way to define the ``best'' solution is to minimize the norm of $\vec{u}$. This is known as minimum energy control.
\end{frame}

\begin{frame}{Minimum Energy Control Solution}
	Notice that $\mathscr{C}$ has a nontrivial nullspace. If we have a particular solution $\vec{u}$ satisfying $\mathscr{C} \vec{u} = \vec{x}^*$ and any vector $\vec{a} \in \mathrm{Null}(\mathscr{C})$, then
	\[
	\mathscr{C}(\vec{u} + \vec{a}) = \mathscr{C}\vec{u} + \mathscr{C}\vec{a} = \mathscr{C}\vec{u} + \vec{0} = \vec{x}^*
	\]
	It turns out that the optimal solution is $\vec{u_0} \perp \mathrm{Null}(\mathscr{C})$.
\end{frame}

\begin{frame}{Minimum Energy Control Proof}
	Let $\vec{u}_0$ be defined as before, and consider a solution vector $\vec{u} = \vec{u}_0 + \vec{a}$ where $\vec{a}$ is in the nullspace of $\mathscr{C}$. We calculate the norm:
	\begin{align*}
	\left| \vec{u} \right| 
	& = \left< \vec{u}, \vec{u} \right> 
	\\ &= \left< \vec{u}_0 + \vec{a}, \vec{u}_0 + \vec{a} \right> 
	\\ &= \left< \vec{u}_0, \vec{u}_0 \right>  + 2\left< \vec{u}_0, \vec{a} \right>  + \left< \vec{a}, \vec{a} \right> 
	\\ &= \left< \vec{u}_0, \vec{u}_0 \right>  + \left< \vec{a}, \vec{a} \right> 
	\\ &= \left| \vec{u}_0 \right| + \left| \vec{a} \right| 
	\end{align*}
	Thus, to minimize the norm of $\vec{u}$ we must minimize the norm of $\vec{a}$, so the optimal solution is $\vec{u} = \vec{u}_0$.
\end{frame}