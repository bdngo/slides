\section{System ID}

\begin{frame}{System ID Parameters}
	(Note: This is basically all of note 7B. )
    We model linear systems with:

    \[
        \vec{x_d}[t+1] = A\vec{x_d}[t] + B\vec{u_d}[t]
    \]
        
    Where 
   	\[
		A = 
		\begin{bmatrix}
			a_{11} & \cdots & a_{1n} \\
			\vdots & \ddots & \vdots \\
			a_{n1} & \cdots & a_{nn}     
		\end{bmatrix}
		B = 
		\begin{bmatrix}
			b_{11} & \cdots & b_{1k} \\
			\vdots & \ddots & \vdots \\
			b_{n1} & \cdots & b_{nk}     
		\end{bmatrix}        
    \]
\end{frame}
	
\begin{frame}{System ID Parameters}
	Suppose we are not given A and B, and wish to determine them with some number of observations. We can treat the matrices as unknown parameters, and attempt to solve for their value. 
	
	Breaking the system down into scalars, we have:
	\[
		\begin{bmatrix}
			x_1[t+1] \\
			\vdots \\
			x_n[t+1] 
		\end{bmatrix}
		=
		\begin{bmatrix}
			a_{11} & \cdots & a_{1n} \\
			\vdots & \ddots & \vdots \\
			a_{n1} & \cdots & a_{nn}     
		\end{bmatrix}
		\begin{bmatrix}
			x_1[t] \\
			\vdots \\
			x_n[t] 
		\end{bmatrix}
		+
		\begin{bmatrix}
			b_{11} & \cdots & b_{1k} \\
			\vdots & \ddots & \vdots \\
			b_{n1} & \cdots & b_{nk}     
		\end{bmatrix}        
		\begin{bmatrix}
			u_1[t] \\
			\vdots \\
			u_k[t] 
		\end{bmatrix}
    \]
\end{frame}

\begin{frame}{Converting to Least Squares}
	Let us examine each row of the previous equation separately. For row $r$, 
	\[
		x_r[t+1]
		=
		\begin{bmatrix}
			a_{r1} & \cdots & a_{rn}   
		\end{bmatrix}
		\begin{bmatrix}
			x_1[t] \\
			\vdots \\
			x_n[t] 
		\end{bmatrix}
		+
		\begin{bmatrix}
			b_{r1} & \cdots & b_{rk} \\  
		\end{bmatrix}        
		\begin{bmatrix}
			u_1[t] \\
			\vdots \\
			u_k[t] 
		\end{bmatrix}
    \]
    Since inner products are symmetric, we can flip this around:
	\[
		x_r[t+1]
		=
		\begin{bmatrix}
			x_1[t] & \cdots & x_n[t] 	
		\end{bmatrix}
		\begin{bmatrix}
			a_{r1} \\ 
			\vdots \\
			a_{rn}   
		\end{bmatrix}
		+   
		\begin{bmatrix}
			u_1[t] & \cdots & u_k[t]
		\end{bmatrix}
		\begin{bmatrix}
			b_{r1} \\ 
			\vdots \\
			b_{rk} 
		\end{bmatrix}     
    \]    
\end{frame}
	
\begin{frame}{Converting to Least Squares}
	Finally, we stack our vectors to get:
	\[
		x_r[t+1]
		=
		\begin{bmatrix}
			x_1[t] & \cdots & x_n[t] & u_1[t] & \cdots & u_k[t]
		\end{bmatrix}
		\begin{bmatrix}
			a_{r1} \\ 
			\vdots \\
			a_{rn} \\
			b_{r1} \\ 
			\vdots \\
			b_{rk} 
		\end{bmatrix}
	\]
	
	Observe that the quantities in the row vector can be observed, while the the quantities in the column vectors are unknown parameters we wish to solve for.
\end{frame}
	
\begin{frame}{System ID Least Squares}
	Unfortunately, the previous equation is not sufficient since we have 1 equation but $n+k$ parameters. Since the equation holds for all times, we can consider all timesteps $0<t\leq T$ to have the following:
	\[
		\small
		\begin{bmatrix}
			x_r[1] \\
			\vdots \\
			x_r[T] \\
		\end{bmatrix}
		=
		\begin{bmatrix}
			x_1[0] & \cdots & x_n[0] & u_1[0] & \cdots & u_k[0] \\
			\vdots & \ddots & \vdots & \vdots & \ddots & \vdots \\
			x_1[T-1] & \cdots & x_n[T-1] & u_1[T-1] & \cdots & u_k[T-1] \\
		\end{bmatrix}
		\begin{bmatrix}
			a_{r1} \\ 
			\vdots \\
			a_{rn} \\
			b_{r1} \\ 
			\vdots \\
			b_{rk} 
		\end{bmatrix}
	\]
	This allows us to solve for our unknowns as long as there are at least $T\geq n+k$ observations using least squares. However, we can make the procedure more efficient if we stack these equations horizontally for each value of $r$.
\end{frame}

\begin{frame}{System ID Least Squares}
	Let us define the following matrices:
	\[
		P 
		= 
		\begin{bmatrix}
			A^T \\ 
			B^T
		\end{bmatrix}
		=
		\begin{bmatrix}
			a_{11} & \cdots & a_{n1} \\
			\vdots & \ddots & \vdots \\
			a_{1n} & \cdots & a_{nn} \\
			b_{11} & \cdots & b_{n1} \\
			\vdots & \ddots & \vdots \\
			b_{1k} & \cdots & b_{nk}     
		\end{bmatrix}  
	\]
\end{frame}

\begin{frame}{System ID Least Squares}
	Let us define the following matrices:
	\begin{align*}
		D
		& =
		\begin{bmatrix}
			\vec{x^T}[0] & \vec{u^T}[0] \\
			\vdots & \vdots \\
			\vec{x^T}[T-1] & \vec{u^T}[T-1] \\
		\end{bmatrix}
		\\ & =
		\begin{bmatrix}
			x_1[0] & \cdots & x_n[0] & u_1[0] & \cdots & u_k[0] \\
			\vdots & \ddots & \vdots & \vdots & \ddots & \vdots \\
			x_1[T-1] & \cdots & x_n[T-1] & u_1[T-1] & \cdots & u_k[T-1] \\
		\end{bmatrix}
	\end{align*}
\end{frame}

\begin{frame}{System ID Least Squares}
	Let us define the following matrices:
	\[
		S
		=
		\begin{bmatrix}
			\vec{x^T}[0] \\
			\vdots \\
			\vec{x^T}[T-1] \\
		\end{bmatrix}
		=
		\begin{bmatrix}
			x_1[1] & \cdots & x_n[1] \\
			\vdots & \ddots & \vdots \\
			x_1[T] & \cdots & x_n[T] \\
		\end{bmatrix}
	\]
\end{frame}

\begin{frame} {System ID Least Squares}
	With these matrices, we finally have
	\[
	DP \approx S
	\]
	Using least squares, we solve
	\[
	P = \left( D^T D \right) ^ {-1} D^T S
	\]
\end{frame}