
\section{Linearization}

\begin{frame}{Linearization}
\begin{itemize}


\item Recall that if we have $\frac{dx}{dt} = \lambda x(t) + bu(t)$ we know a solution to this is: $$x(t) = x(0)e^{\lambda t} + \int_0^t \! e^{\lambda(t - \tau)}u(\tau)\ d\tau$$
\item What if we had $\frac{dx}{dt} = f(x(t)) + bu(t)$, where $f$ is nonlinear (e.g $sin$)? \pause \\
\item Big Picture: linearize $f$ around an operating point and then treat it as a linear function in a small neighborhood of that point.\\
\item Why linearization? \pause
\\ It allows you to treat the system as a linear one, which is very helpful as linear ODE are (usually) much easier to solve!
\end{itemize}
\end{frame}
\begin{frame}{Linearizing a Single-Variable Function}
\begin{itemize}
\item Suppose we have $f(x)$ that is a non linear function. We can use a first order Taylor polynomial to linearize the function, this is equivalent to finding the slope of the tangent line of $f(x)$ at a particular point.
\item From calculus: $f(x) \approx f(x^*) + f'(x^*)(x - x^*)$.
\item As long as we are within some (very small) $\delta$ neighborhood of $x^*$ the linearization is valid.
\item Example: Linearize $f(x) = 3e^{x^2 + 2}$ around $x^*$ 
    \pause
\item Solution: \\
    $f(x^*) = 3e^{(x^{*})^2 + 2}$\\
$f'(x) = 3e^{x^2+2}(2x) = 6xe^{x^2+2}$\\
        $f'(x^*) = 6x^*e^{(x^{*})^2+2}$\\
Therefore :
        $f(x) \approx 3e^{(x^{*})^2 + 2} +  6x^*e^{(x^{*})^2+2}(x-x^*) $
\end{itemize}

\end{frame}
\begin{frame}{Linearizing Steps for $\frac{dx(t)}{dt} = f(x(t)) + bu(t)$}
\begin{enumerate}[(i)]
\item Choose, or you may be given, a DC input point. That is, a point $u^* \equiv u(t)$ that is constant with time.\pause \\
\item Find a DC operating point, $x^* \equiv x(t)$. That is, solve $\frac{dx^*}{dt} = f(x^*) + bu^*$. Notice that this boils down to finding an $x^*$ such that $f(x^*) + bu^* = 0 $. \pause \\
\item Define $x_l(t) = x(t) - x^*$ and $u_l(t) = u(t) - u^*$, and re-write the ODE in terms of $ x_l(t)$ and $ u_l(t)$. By plugging in you get: $\frac{dx_l(t)}{dt} = f(x_l(t) + x^*) + b(u_l(t)+ u^*)$ \pause \\
\item It is ok to assume at this point that $u_l(t)$ is small, that means that the $u(t)$ in step 1 does not deviate too much from $u^*$.\pause \\
\item Linearize the ODE: $f(x_l(t) + x^*) \approx f(x^*) + f'(x^*)x_l(t)$. Here we assume that $x_l(t)$ is also small. This is something that we will need to verify in the next step!
\end{enumerate}
\end{frame}
\begin{frame}{Linearizing Steps for $\frac{dx(t)}{dt} = f(x(t)) + bu(t)$ (Continued)}
\begin{enumerate}[(vi)]
\item Plug (vi) back into  (iii) and we obtain : $\frac{dx_l(t)}{dt} \approx f'(x^*)f(x_l(t)) + \cancel{f(x^*)} + bu_l(t)+ \cancel{bu^*} = f'(x^*)f(x_l(t))+ bu_l(t)$
\item Verify the linearization! \\
How do we know if the linearization is valid? \pause Well, if we have $\frac{dx_l(t)}{dt} = \lambda x_l(t) + bu(t)$ we know the solution doesn't blow up if $\lambda < 0$ as we will have a term $e^{\lambda t}$.
\\ This means that we want $m = f'(x^*) <$  0. 
\end{enumerate} \pause
So what do we do if $m>0 $? \pause \\
We need to go back and change our DC operating point $x^*$
\end{frame}
\begin{frame}{Practice Problem}
Linearize $\frac{dx(t)}{dt} = \cos(x(t)) + bu(t)$ about $u^* = 0$. \pause \\
%Show this hint after a while if people are stuck
\textit{Hint:} $\cos(x^*) = 0$ has multiple solutions, which means that we can find numerous DC operating points, can you guess which one would result in a stable system ? 
\end{frame}
\begin{frame}{Practice Problem Solution}
\begin{enumerate}[(i)]
\item We were given the DC input, $u^* = 0$ \pause \\
\item $\cos(x^*)$ = 0, which means that $x^* = k\frac{\pi}{2}$ for $k \in \{...-2,-1,1,2,...\}$. We will choose $x^* = \frac{\pi}{2}$ \pause \\
\item Let $x_l(t) = x(t) - \frac{\pi}{2}$ and $u_l(t) = u(t) - 0$.
By plugging in we get: $\frac{dx_l(t)}{dt} = \cos(x_l(t) + \frac{\pi}{2}) + u_l(t)$ \pause \\
\item We assume that $u_l(t)$ is small.\pause \\
\item Linearize the ODE: $\cos(x_l(t) + \frac{\pi}{2}) \approx \cos(\frac{\pi}{2}) -\sin(\frac{\pi}{2})x_l(t)$. \pause
\item Plug (v) back into ODE: 
$\frac{dx_l(t)}{dt} = -x_l(t) + u_l(t)$ 
\end{enumerate}
\pause 
We see that our assumption that  $x_l(t)$ is small is indeed satisfied as we will have a $e^{-t}$ term in the solution which means that $x_l(t)$ will decay.\\ \pause
What if we had chosen a different DC Operating point, say $-\frac{\pi}{2}$? When we linearize the system we see that the solution will "explode" around that particular DC operating point.
\end{frame}

\begin{frame}{Linearization of Vector Functions}
What if we had $\frac{d\vec{x}}{dt} = \vec{f}(\vec{x}, \vec{u})$ ? We will see that the process is very similar to the scalar case we just analyzed!\\ \pause
First, let's see how to linearize $\vec{f}(\vec{x})$ around a DC operating point $\vec{x}^*$. Where $\vec{f} \in \mathbb R^{n \times 1}$ is a vector of scalar functions. \\\pause
The idea is to linearize individually each one of the $f_i$ around the DC operating point. \\\pause
For example: $f_1 \approx f_1(\vec{x}^*) + \frac{\partial f_1}{\partial x_1}(\vec{x}^*)(x_1 - x_1^*) + ... +  \frac{\partial f_n}{\partial x_1}(\vec{x}^*)(x_n - x_n^*)$\\\pause
Repeating this for all $n$ functions in $\vec{f}$ we see we get a system of scalar, linearized, multivariate functions which makes you think, wouldn't it be nice to express it in a shorthand matrix notation?
\end{frame}

\begin{frame}{Jacobian Matrix}
We can use the Jacobian to express everything nicely and neatly. \\
The Jacobian is the name given to the matrix of partial derivatives of $\vec{f}$, and it is denoted by $J_{\vec{x}}$ or $\nabla_{\vec{x}}\vec{f}$. \\ \pause
Continuing from the previous slide we have: \\
\[
\begin{bmatrix} f_1(\vec{x}) \\ \vdots \\f_n(\vec{x}) \end{bmatrix} \approx 
\begin{bmatrix} f_1(\vec{x}^*) \\ \vdots \\f_n(\vec{x}^*) \end{bmatrix} + 
\begin{bmatrix} \frac{\partial f_1}{\partial x_1}(\vec{x}^*)& \hdots& \frac{\partial f_1}{\partial x_n}(\vec{x}^*) \\ \vdots & \ddots &\vdots \\ \frac{\partial f_n}{\partial x_1}(\vec{x}^*)& \hdots &  \frac{\partial f_n}{\partial x_n}(\vec{x}^*) \end{bmatrix}
(\vec{x} - \vec{x}^*)
\]
\\

\end{frame}

\begin{frame}{Linearization with Jacobians Example}
\begin{align*}
    \text{\hspace{-3ex}Linearize}  \hspace{0.2 cm} 
\vec{f}(\vec{x}(t))  = 
\begin{bmatrix}
\sin(x_1(t)\times x_2(t)) + 2x_1(t)x_3^2(t)\\
x_3(t)\cos(x_2(t)) + \frac{x_1(t)}{x_3(t)}\\
x_1(t) + 2x_3(t)x_2^3(t)
\end{bmatrix} 
\text{about} \hspace{0.2 cm} \vec{x}^* = \begin{bmatrix}
0\\ 2 \pi \\ \frac{2 \pi}{3}
\end{bmatrix} 
\end{align*}
\end{frame}

\begin{frame}{Solutions}
Find the Jacobian: 
\small
\[
    \hspace{-2ex}\begin{bmatrix}
x_2(t)\cos(x_1(t)\times x_2(t)) + 2x_3^2(t)& x_1(t)\cos(x_1(t)\times x_2(t)) & 4x_1(t)x_3(t)\\ 
\frac{1}{x_3(t)}& -x_3(t)\sin(x_2(t)) &\cos(x_2(t)) - \frac{x_1(t)}{x_3^2(t)}  \\ 
1 & 6x_3(t)x_2^2(t)& 2 x_2^3(t) 
\end{bmatrix}
\]
\normalsize
Evaluate the Jacobian about $\vec{x}^*$:
\[
\begin{bmatrix}
5 \pi & 0 & 0\\
\frac{2 \pi }{3}& 0 & 1\\
1 & 36 \pi^3 & 16 \pi^3
\end{bmatrix}
\]
\\
Linearize:
\begin{equation*}
\vec{f}(\vec{x}(t)) \approx 
\begin{bmatrix}
0 \\ \frac{3\pi }{4} \\ 24 \pi^4
\end{bmatrix} +
\begin{bmatrix}
5 \pi & 0 & 0\\
\frac{2 \pi }{3}& 0 & 1\\
1 & 36 \pi^3 & 16 \pi^3
\end{bmatrix}
\begin{bmatrix}
x_1(t) - 0 \\ x_2(t) - \frac{3\pi }{4} \\ x_3(t) - 24 \pi^4
\end{bmatrix}
\end{equation*}

\end{frame}

\begin{frame}{Steps to Linearize Vector ODE Systems}
To linearize $\frac{d\vec{x}}{dt} = \vec{f}(\vec{x}(t), \vec{u}(t))$ we use a similar procedure as we did for the scalar case. \pause 
\begin{enumerate}[(i)]
%	    \item If you're not given a DC input $\vec{u}^*$, determine one. \pause \\
\item Solve $\vec{f}(\vec{x}^*, \vec{u}^*) = \vec{0}$ to determine the equilibrium point. \pause \\
\item Let $\vec{x}_l(t) = \vec{x}(t) - \vec{x}^*$ and $\vec{u}_l(t) = \vec{u}(t) - \vec{u}^*$ \pause \\
\item Linearize $\vec{f}(\vec{x}, \vec{u})$ about $(\vec{x}^*, \vec{u}^*)$. That is: $\vec{f}(\vec{x}(t), \vec{u}(t)) \approx \vec{f}(\vec{x}^*, \vec{u}^*) + J_{\vec{x}}\vec{x}_l(t) + J_{\vec{u}}\vec{u}_l(t)$ \pause \\
\item Plug (iv) back into the ODE: $\frac{d\vec{x}}{dt} \approx \cancel{\vec{f}(\vec{x}^*, \vec{u}^*)} + J_{\vec{x}}\vec{x}_l(t) + J_{\vec{u}}\vec{u}_l(t)$
\end{enumerate}

\end{frame}

% \begin{frame}{Linearizing Vector ODE Systems Example}
% Given a DC input $u^* = 1$, linearize: 
% \[
% \frac{d\vec{x}(t)}{dt} = \begin{bmatrix}
% x_1^2(t) - x_2(t)u(t)\\
% x_2^2(t)(1 + x_1(t)) + \sin(\pi x_1(t)u(t))
% \end{bmatrix}
% \]

% \end{frame}

% \begin{frame}{Solutions}
% Again, we will do this in steps: 
% \begin{enumerate}[(i)]
% \item We are given $u^* = 1$ \pause
% \item We need to find a DC operating point, this means solving the following system of equations:
% \begin{align}
% x_1^{*2}- x_2^*u^* = 0 \\
% x_2^{*2}(x_1^* + 1) + \sin(\pi x_1^*u^*) = 0
% \end{align} 
% The solution is $x_1^* = -1$ and $x_2^* = 1$. \pause \\
% \item Let $\vec{x}_l(t) = \vec{x}(t) - \vec{x}^*$ and $\vec{u}_l(t) = \vec{u}(t) - \vec{u}^*$ \pause \\
% \item Linearize, 
% \[ \vec{f}(\vec{x}(t), u(t)) \approx 
% \vec{f}(\vec{x}^*, 1) + \begin{bmatrix}
% -2 & -1 \\
% 1 - \pi & 0 
% \end{bmatrix} 
% \vec{x}_l(t) + \begin{bmatrix}
% -1 \\ \pi
% \end{bmatrix}
% u_l(t)
% \]



% \end{enumerate}
% \end{frame}

% \begin{frame}{Solutions Continued}
% \begin{enumerate}[(v)]
% \item Substitute linear approximation back into the system,
% \[ \frac{d\vec{x}(t)}{dt} \approx 
% \cancel{\vec{f}(\vec{x}^*, 1)} + \begin{bmatrix}
% -2 & -1 \\
% 1 - \pi & 0 
% \end{bmatrix} 
% \vec{x}_l(t) + \begin{bmatrix}
% -1 \\ \pi
% \end{bmatrix}
% u_l(t)
% \]
% \end{enumerate}
% \end{frame}


